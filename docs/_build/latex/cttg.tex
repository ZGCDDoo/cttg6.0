%% Generated by Sphinx.
\def\sphinxdocclass{report}
\documentclass[letterpaper,10pt,english]{sphinxmanual}
\ifdefined\pdfpxdimen
   \let\sphinxpxdimen\pdfpxdimen\else\newdimen\sphinxpxdimen
\fi \sphinxpxdimen=.75bp\relax

\usepackage[utf8]{inputenc}
\ifdefined\DeclareUnicodeCharacter
 \ifdefined\DeclareUnicodeCharacterAsOptional
  \DeclareUnicodeCharacter{"00A0}{\nobreakspace}
  \DeclareUnicodeCharacter{"2500}{\sphinxunichar{2500}}
  \DeclareUnicodeCharacter{"2502}{\sphinxunichar{2502}}
  \DeclareUnicodeCharacter{"2514}{\sphinxunichar{2514}}
  \DeclareUnicodeCharacter{"251C}{\sphinxunichar{251C}}
  \DeclareUnicodeCharacter{"2572}{\textbackslash}
 \else
  \DeclareUnicodeCharacter{00A0}{\nobreakspace}
  \DeclareUnicodeCharacter{2500}{\sphinxunichar{2500}}
  \DeclareUnicodeCharacter{2502}{\sphinxunichar{2502}}
  \DeclareUnicodeCharacter{2514}{\sphinxunichar{2514}}
  \DeclareUnicodeCharacter{251C}{\sphinxunichar{251C}}
  \DeclareUnicodeCharacter{2572}{\textbackslash}
 \fi
\fi
\usepackage{cmap}
\usepackage[T1]{fontenc}
\usepackage{amsmath,amssymb,amstext}
\usepackage{babel}
\usepackage{times}
\usepackage[Bjarne]{fncychap}
\usepackage[dontkeepoldnames]{sphinx}

\usepackage{geometry}

% Include hyperref last.
\usepackage{hyperref}
% Fix anchor placement for figures with captions.
\usepackage{hypcap}% it must be loaded after hyperref.
% Set up styles of URL: it should be placed after hyperref.
\urlstyle{same}

\addto\captionsenglish{\renewcommand{\figurename}{Fig.}}
\addto\captionsenglish{\renewcommand{\tablename}{Table}}
\addto\captionsenglish{\renewcommand{\literalblockname}{Listing}}

\addto\captionsenglish{\renewcommand{\literalblockcontinuedname}{continued from previous page}}
\addto\captionsenglish{\renewcommand{\literalblockcontinuesname}{continues on next page}}

\addto\extrasenglish{\def\pageautorefname{page}}

\setcounter{tocdepth}{0}



\title{cttg Documentation}
\date{Jun 11, 2018}
\release{5.1.1}
\author{Charles-David Hebert, Maxime Charlebois, Patrick Semon}
\newcommand{\sphinxlogo}{\vbox{}}
\renewcommand{\releasename}{Release}
\makeindex

\begin{document}

\maketitle
\sphinxtableofcontents
\phantomsection\label{\detokenize{index::doc}}



\chapter{Introduction}
\label{\detokenize{intro:introduction}}\label{\detokenize{intro:welcome-to-cttg-s-documentation}}\label{\detokenize{intro::doc}}
cttg stands for \sphinxstyleemphasis{Continuous time Tremblay group}
and regroups two main Continuous-time quantum monte-carlo algorithms, namely CT-INT and CT-AUX.


\section{Warning: Bad Docs}
\label{\detokenize{intro:warning-bad-docs}}
The documentation has only been started very recently. It is thus full of language errors, probably wrong at certain places and of poor quality.
However, the quality will increase overtime.


\section{Conventions}
\label{\detokenize{intro:conventions}}

\subsection{Comments}
\label{\detokenize{intro:comments}}\begin{description}
\item[{We use the convention "\$" for the start of a shell command and "\#" for commenting shell commands, etc.}] \leavevmode\begin{description}
\item[{Ex:}] \leavevmode
\$ cd path/to/thing    \# change to the path of thing.

\end{description}

\end{description}


\subsection{Executable names}
\label{\detokenize{intro:executable-names}}
When not specified, the algorithm is CT-INT. if there is "aux" in the name, then CT-AUX.
If "sub" in name, then submatrix algorithm, else normal fast-update scheme.


\subsection{Hamiltonian}
\label{\detokenize{intro:hamiltonian}}\begin{equation*}
\begin{split}H = + \sum_{ij} t_{ij} + U \sum_{i} n_{i \uparrow} n_{i \downarrow}\end{split}
\end{equation*}
Thus, for the cuprates, the convention of the program is
\begin{itemize}
\item {} 
t = -1.0

\item {} 
t' = 0.3

\item {} 
t'' = -0.2

\end{itemize}


\chapter{Installation}
\label{\detokenize{installation:installation}}\label{\detokenize{installation::doc}}\label{\detokenize{installation:id1}}
\sphinxstylestrong{Note :}
If build problems,
please remove the build directory if it exists, then retry :
\begin{quote}

\$ rm -rf build
\end{quote}


\section{Dependencies}
\label{\detokenize{installation:dependencies}}\begin{enumerate}
\item {} 
Armadillo

\item {} 
boost (mpi, serialization, filesystem, system)

\end{enumerate}


\section{Pre-Steps}
\label{\detokenize{installation:pre-steps}}\begin{enumerate}
\item {} 
Make sure you have a "bin" directory in your home folder

\item {} 
Append the bin folder to your path. Add the following line to your \textasciitilde{}/.bashrc:  export PATH="\$PATH:\textasciitilde{}/bin"

\item {} 
\$ source \textasciitilde{}/.bashrc

\end{enumerate}


\section{Linux (Ubuntu 16.04)}
\label{\detokenize{installation:linux-ubuntu-16-04}}
This installation procedure should work for many recent Linux flavors. For the following
we present the instructions specific for Ubuntu or derivatives.
\begin{enumerate}
\item {} \begin{description}
\item[{Install the Dependencies}] \leavevmode
\$ sudo apt-get install libarmadillo-dev libboost-all-dev cmake

\end{description}

\item {} 
\begin{DUlineblock}{0em}
\item[] \$ mkdir build \&\& cd build \&\& cmake -DTEST=OFF .. \&\& make -j NUMBER\_OF\_CORES install
\item[] \# replace NUMBER\_OF\_CORE by say = 4
\end{DUlineblock}

\end{enumerate}


\section{Mac}
\label{\detokenize{installation:mac}}
This has been tested once. MPI not yet supported.
\begin{enumerate}
\item {} \begin{description}
\item[{Install the Dependencies}] \leavevmode
\$ brew install armadillo boost

\end{description}

\item {} 
\begin{DUlineblock}{0em}
\item[] \$ mkdir build \&\& cd build \&\& cmake -DHOME=OFF -DMAC=ON  .. \&\& make -j NUMBER\_OF\_CORES install
\item[] \# replace NUMBER\_OF\_CORE by say = 4
\end{DUlineblock}

\end{enumerate}


\section{Mp2}
\label{\detokenize{installation:mp2}}\begin{enumerate}
\item {} 
\$ module reset

\item {} 
\$ module load cmake/3.6.1  gcc/6.1.0  intel64/17.4  boost64/1.65.1\_intel17 openmpi/1.8.4\_intel17  armadillo/8.300.0

\item {} 
\$ mkdir build \&\& cd build \&\& cmake -DHOME=OFF -DMP2=ON -DMPI\_BUILD=ON .. \&\& make install

\end{enumerate}


\section{Graham and Ceder}
\label{\detokenize{installation:graham-and-ceder}}\begin{enumerate}
\item {} 
\$ module reset

\item {} 
\$ module load nixpkgs/16.09  gcc/5.4.0 armadillo boost-mpi

\item {} 
\begin{DUlineblock}{0em}
\item[] \$ mkdir build \&\& cd build \&\& \textbackslash{}
\item[] cmake -DHOME=OFF -DGRAHAM=ON -DMPI\_BUILD=ON  .. \&\& make install
\end{DUlineblock}

\end{enumerate}
\phantomsection\label{\detokenize{tutorial:tutorial}}

\chapter{Tutorial}
\label{\detokenize{tutorial::doc}}\label{\detokenize{tutorial:id1}}
For Mp2 and Graham, use the "scractch" directories. If build problems,
please remove the build directory if it exists, then retry :
\begin{quote}

\$ rm -rf build
\end{quote}


\section{Graham: Tutorial 1}
\label{\detokenize{tutorial:graham-tutorial-1}}\begin{enumerate}
\item {} 
\begin{DUlineblock}{0em}
\item[] Connect to Graham:
\item[] \$ ssh -X "user"@graham.computecanada.ca \# where "user" is your compute canada/Mp2 Username
\end{DUlineblock}

\item {} 
Ensure you have done the Pre-Steps described in {\hyperref[\detokenize{installation:installation}]{\sphinxcrossref{\DUrole{std,std-ref}{Installation}}}}.

\item {} 
\$ salloc  --time=01:00:00 --ntasks=1 --mem-per-cpu=4000

\item {} 
Follow th Graham Procedure in {\hyperref[\detokenize{installation:installation}]{\sphinxcrossref{\DUrole{std,std-ref}{Installation}}}}.

\item {} 
\$ module reset

\item {} 
\$ module load nixpkgs/16.09  gcc/5.4.0 armadillo boost-mpi

\item {} 
\$ cd ../examples/CDMFT

\item {} 
\$ cdmft\_square4x4 params 1

\end{enumerate}

This is  only an example, as the results will be wrong, because the
Measurement time and the Updates bewteen measurements is to low.
To get sensible results, set MEASUREMENT\_TIME to say 10, and UPDATES\_MEAS
to 100.


\section{Mp2: Tutorial 1}
\label{\detokenize{tutorial:mp2-tutorial-1}}\begin{enumerate}
\item {} 
Connect to Mp2:

\item {} 
Ensure you have done the Pre-Steps described in {\hyperref[\detokenize{installation:installation}]{\sphinxcrossref{\DUrole{std,std-ref}{Installation}}}}.

\item {} 
Follow th Mp2 Procedure in {\hyperref[\detokenize{installation:installation}]{\sphinxcrossref{\DUrole{std,std-ref}{Installation}}}}.

\item {} 
\$ cd /path/to/cttg3.0/examples/CDMFT

\item {} 
\$ cp ../Mp2/scriptMp2.pbs ./

\item {} 
Change the walltime to 01:00:00 and the queue: "qwork" -\textgreater{} "qtest"

\item {} 
Change "myExe=dmft" to "myExe=cdmft\_square4x4"

\item {} 
qsub scriptMp2.pbs

\end{enumerate}

The simulation should launch fairly quickly (in the hour).

This is  only an example, as the results will be wrong, because the
Measurement time and the Updates bewteen measurements is to low.
To get sensible results, set MEASUREMENT\_TIME to say 10, and UPDATES\_MEAS
to 100.


\section{Home}
\label{\detokenize{tutorial:home}}

\subsection{Home: Tutorial 1 = dmft}
\label{\detokenize{tutorial:home-tutorial-1-dmft}}
To Come. For now, go into the "examples folder", select your installation and run the bash script.
If there are some problems, then install "dos2unix" and run it on the bash files.
\begin{quote}

\$dos2unix runCDMFT
\end{quote}
\begin{description}
\item[{Ex:}] \leavevmode
If you installed on your computer
1. \$ cd examples/Home
2. \$ bash runCDMFT.sh \# in fact runs dmft

\end{description}


\subsection{Home: tutorial 2 = cdmft\_square4x4}
\label{\detokenize{tutorial:home-tutorial-2-cdmft-square4x4}}\begin{enumerate}
\item {} 
\$ cd examples/CDMFT

\item {} \begin{description}
\item[{copy the script file, for example if on Home:}] \leavevmode
\$ cp ../Home/runCDMFT ./

\end{description}

\item {} \begin{description}
\item[{in "runCDMFT", replace the line:}] \leavevmode
myExe=dmft  -\textgreater{} myExe=cdmft\_square4x4

\end{description}

\item {} 
\$ bash runCDMFT

\end{enumerate}


\subsection{Launching a simulation}
\label{\detokenize{tutorial:launching-a-simulation}}
To launch a simulation, you need three files:
\begin{enumerate}
\item {} 
a script file, dependant on the platform (home, mp2, graham-cedar)

\item {} 
a "params" file (Ex: params1.json)

\item {} 
a "hyb" file (Ex: hyb1Up.dat)

\end{enumerate}


\section{When to use which algorithm}
\label{\detokenize{tutorial:when-to-use-which-algorithm}}
CT-Aux and CT-INT behave in a similar manner, and from my experience, one is not much faster than the other.

Cocerning Submatrix Updates, the codes "...\_sub" should be used when the expansion order is high, say k\textgreater{}400
For k\textasciitilde{}\textless{}200, the algorithm may be slower.
\begin{description}
\item[{Ex:}] \leavevmode
cdmft\_square4x4 for k \textless{} 400
cdmft\_square4x4\_sub for k \textgreater{} 400

\end{description}


\chapter{The params file}
\label{\detokenize{params:the-params-file}}\label{\detokenize{params::doc}}

\section{Fast-Update Scheme}
\label{\detokenize{params:fast-update-scheme}}

\subsection{Main Paremeters}
\label{\detokenize{params:main-paremeters}}
The following parameters are the ones that the user has to change and need to understand. There is a bit of unconsistency, to be corrected
(some are lower case, while other are upper case.)
\begin{quote}
\begin{description}
\item[{modelType}] \leavevmode
the model to run the simulations, can be one of the following:
* "SIAM\_Square"
* "Square2x2"
* "Square4x4"
* "Triangle2x2"

\item[{solver}] \leavevmode
the solver to use, can be one of the following:


\begin{savenotes}\sphinxattablestart
\centering
\begin{tabulary}{\linewidth}[t]{|T|T|}
\hline
\sphinxstylethead{\sphinxstyletheadfamily 
cttg
\unskip}\relax &\sphinxstylethead{\sphinxstyletheadfamily 
cttg\_sub
\unskip}\relax \\
\hline
"Int"
&
"IntSub"
\\
\hline
"Aux"
&
"IntAux"
\\
\hline
\end{tabulary}
\par
\sphinxattableend\end{savenotes}

\item[{SEED}] \leavevmode
the seed for the random number generator.

\item[{beta}] \leavevmode
inverse temperature

\item[{EGreen}] \leavevmode
the cutoff of matsubara frequencies in energy for the measurement of the green fucntions, 100 is fine.

\item[{NTAU}] \leavevmode
the time discretization for G(tau) and for binning measurements. Normally
a value of 1000 is sufficient, but, for low temperatures and big EGreen,
a higher value is neccessary to get unbiaised results. I recommend NTAU \textasciitilde{} beta 150.
If you so wish, I take a minimum value of NTau = beta*125 in the code. I thus take
only NTau if it is bigger than this minimum value.


\begin{savenotes}\sphinxattablestart
\centering
\begin{tabulary}{\linewidth}[t]{|T|T|}
\hline
\sphinxstylethead{\sphinxstyletheadfamily 
beta
\unskip}\relax &\sphinxstylethead{\sphinxstyletheadfamily 
NTAU
\unskip}\relax \\
\hline
10
&
1500
\\
\hline
50
&
7500
\\
\hline
\end{tabulary}
\par
\sphinxattableend\end{savenotes}

\item[{UPDATESMEAS}] \leavevmode
The numbre of Updates proposed bewteen each measurement.
This value should be approximately equal to the average expansion order.
Updates proposed bewteen measurements = UPDATESMEAS. I recommend UPD \textless{} k.


\begin{savenotes}\sphinxattablestart
\centering
\begin{tabulary}{\linewidth}[t]{|T|T|}
\hline
\sphinxstylethead{\sphinxstyletheadfamily 
k
\unskip}\relax &\sphinxstylethead{\sphinxstyletheadfamily 
UPD
\unskip}\relax \\
\hline
200
&
100
\\
\hline
500
&
250
\\
\hline
800
&
300
\\
\hline
1000
&
350
\\
\hline
1500
&
400
\\
\hline
\end{tabulary}
\par
\sphinxattableend\end{savenotes}

\item[{THERMALIZATION\_TIME}] \leavevmode
the time in minutes for which each processor will thermalize. It is difficult to give a good
optimal value. I would say, \textasciitilde{}10\% of the measurement time.

\item[{MEASUREMENT\_TIME}] \leavevmode
The time in minutes each processor measures. For Mp2, 1 node:


\begin{savenotes}\sphinxattablestart
\centering
\begin{tabulary}{\linewidth}[t]{|T|T|}
\hline
\sphinxstylethead{\sphinxstyletheadfamily 
k
\unskip}\relax &\sphinxstylethead{\sphinxstyletheadfamily 
MT
\unskip}\relax \\
\hline
200
&
5
\\
\hline
500
&
20 try submatrix
\\
\hline
800
&
35 try submatrix
\\
\hline
1000
&
50 try submatrix
\\
\hline
1500
&
90 try submatrix
\\
\hline
\end{tabulary}
\par
\sphinxattableend\end{savenotes}

\item[{WEIGHTSR, WEIGHTSI}] \leavevmode
The real and imaginary part of w in the following:
hyb\_\{n+1\} = w*hyb\_n + (1-w)*hyb\_\{n+1\}

\item[{N\_T\_INV}] \leavevmode
The number of translational invariance measurements to take for ONE given configuration. 5 is a good value. Reduces noise for filling and docc.
Do not put a too big value.

\item[{ESelfCon}] \leavevmode
The cut off in energy to do the SelfConsistency

\item[{n}] \leavevmode
If this parameter is in the params file, than the program will change the chemical potential to attain the given value.

\item[{S}] \leavevmode
If n is given in the params file, then S should also be given. It controls the change of chemical potentiel
according to a newton method. \textasciitilde{}1 is ok.

\end{description}
\end{quote}


\subsection{Implementation detailed Parameters}
\label{\detokenize{params:implementation-detailed-parameters}}
These parameters can be left at their current value, i.e they are implementation details.
Make sure you understand what you are doing before changing them.
\begin{quote}
\begin{description}
\item[{CLEANUPDATE}] \leavevmode
Specifies when to perform a clean update. Ex, if =100, than at each
100 measures, a cleanupdate will be performed. 100 is a good number.
does not substantially influence the simulation, except if this number is to low or to high.

\item[{K}] \leavevmode
The value of the K parameter of CT-Aux. Influences the acceptance rate and the expansion order
1 seems a reasanable value

\item[{delta}] \leavevmode
The value of the delta parameter of CT-INT. Influences the acceptance rate and the expansion order
\textasciitilde{}0.01 seems a reasanable value

\item[{THERM\_FROM\_CONFIG}] \leavevmode
if true, there will be no thermalization, the last saved configuartion will be loaded
and the measurements will start. Not really tested yet. So the default is false.
André-Marie argues that it is better to thermalize each time.

\end{description}
\end{quote}


\section{Submatrix Update Scheme}
\label{\detokenize{params:submatrix-update-scheme}}\begin{quote}
\begin{description}
\item[{KMAX\_UPD}] \leavevmode
the maximum number of updates proposed for each internal iteration (implementation details).
This parameter should be \textasciitilde{}100 on MP2, \textasciitilde{}125 on Graham, and has a different optimal value for different architectures.

\end{description}
\end{quote}



\renewcommand{\indexname}{Index}
\printindex
\end{document}